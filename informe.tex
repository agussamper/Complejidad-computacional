\documentclass{article}
\usepackage[spanish]{babel}
\usepackage{graphicx} % Required for inserting images
\usepackage{lipsum}
\usepackage[a4paper, total={6in, 8in}]{geometry}
\usepackage{setspace}
\usepackage[utf8]{inputenc}
\usepackage{csquotes}
\usepackage{amssymb,amsmath,amsthm, amsfonts}


\setlength{\parskip}{2pt}

\spacing{1.3}

\newtheorem{theorem}{Teorema}
\newtheorem{problem}{Problema}

\title{Complejidad computacional}
\author{Agustín Samper, Facundo Llaudet y Guillermo Pereyra}
\date{November 2023}

\begin{document}
\maketitle
La mayoría de los problemas que consideramos en la materia son de 
carácter general, aplicados a todos los miembros de alguna familia de
grafos o dígrafos. Estos problemas pueden ser clasificados dependiendo de su nivel de dificultad.

La dificultad de un problema depende del tiempo empleado para
devolver una respuesta dado un input en específico.

Estos problemas son resueltos mediantes algoritmos, los cuales
los definimos como procedimientos computacionales bien definidos que 
aceptan cualquier instancia del problema como entrada y retorna una
solución al problema como salida. Por ejemplo,
el algoritmo de Prim acepta como entrada un grafo ponderado y retorna un
árbol óptimo.

Por instancia de un problema, nos referimos al
problema aplicado a un miembro específico de la familia. Por ejemplo,
una instancia del problema del árbol recubridor de peso mínimo es el
problema de encontrar un árbol óptimo (árbol recubridor de peso mínimo)
para un grafo ponderado particular.

Entonces, cuando hablamos del nivel de dificultad de un problema, nos
referimos a la complejidad computacional de un algoritmo que lo resuelve,
al número de pasos básicos computacionales (como operaciones aritméticas y comparaciones)
requeridas para su ejecución. Este número depende claramente del tamaño y de la 
naturaleza de la entrada.

Diremos que un algoritmo es polinomial cuando el número de
operaciones que efectúa está acotado por una función polinomial en
el tamaño de su entrada, el algoritmo es llamado \textit{algoritmo
de tiempo polinomial}. Además, está calificado como \textit{tiempo lineal}
si el polinomio es una función lineal, \textit{tiempo cuadrático} si 
es una función cuadrática y así sucesivamente.

\section*{La clase $\mathcal{P}$}
La importancia de algoritmos de tiempo polinomial es que estos algoritmos
generalmente son computacionalmente realizables, incluso para grandes
tamaños de entrada.

El tipo de problemas resolubles mediante algoritmos de tiempo polinómico
lo denotaremos como problemas de clase $\mathcal{P}$.

Un ejemplo de algoritmo en tiempo polinomial (clase $\mathcal{P}$)
es BFS (breadth-first search) para el cual, cada arista es examinada por una posible
inclusión en el árbol solamente dos veces. Otro ejemplo es DFS (depth-first serach)
donde ocurre lo mismo que en BFS. Por lo tanto, cualquiera de estos son
lineales en el número de aristas.

En contraste a los algoritmos de tiempo polinomial, los
algoritmos cuya complejidad es exponencial en el tamaño de su
entrada tienen tiempos de ejecución, los cuales los hacen inutilizables,
incluso para entradas de un tamaño moderado. Por ejemplo, un algoritmo
el cual verifica si dos grafos con n vértices son isomorfos o no 
considerando todas las n! biyecciones entre sus conjuntos de vértices es
realizable solo para pequeños valores de n (específicamente no mayor a 20).

Estos algoritmos dan lugar a una nueva clase de problemas que llamaremos
problemas de clase $\mathcal{NP}$.

\section*{Las Clases $\mathcal{NP}$ y co-$\mathcal{NP}$}
Necesitamos primero definir que un \textit{problema de decisión}
es una pregunta cuya respuesta es 'sí' o 'no'. Tal problema pertenece
a la clase $\mathcal{P}$ si existe un algoritmo de tiempo
polinomial que resuelve cualquier instancia del problema en tiempo
polinomial. En cambio, dada una instancia
cualquiera del problema cuya respuesta es 'sí', si existe alguna forma de 
certificar que la afirmación es válida en tiempo polinomial, entonces este problema
pertenece a la clase $\mathcal{NP}$; tal certificado decimos
que es un certificado suficiente. Análogamente,
un problema pertenecerá a la clase co-$\mathcal{NP}$ si para
una instancia cualquiera del problema cuya respuesta es 'no', existe un
certificado suficiente.

Es inmediato ver que 
$\mathcal{P}$ $\subseteq$ \textit{$\mathcal{NP}$},
pues por definición, para un problema $\mathcal{P}$ existe
un algoritmo que resuelve en tiempo polinomial cualquier instancia
del problema, en particular lo hará para aquellas cuya respuesta es
'sí'. Siguiendo la misma lógica para las instancias cuyas respuestas
es 'no', llegamos a la conclusión que $\mathcal{P}$
$\subseteq$ co$-\mathcal{NP}$. Entonces, podemos
ver que:
\begin{align*}
    \mathcal{P} \subseteq \mathcal{NP}
    \cap co-\mathcal{NP}
\end{align*}

Consideremos, por ejemplo, el problema de determinar
si un grafo es bipartito. Este problema de decisión pertenece
a $\mathcal{NP}$, ya que si existe una bipartición válida, esta es un
certificado suficiente de que el grafo es bipartito, en efecto,
dada una bipartición $(X,Y)$ de un grafo $G$, es
suficiente con verificar que cada vértice de $G$
tiene un extremo en $X$ y otro en $Y$ para probar que $G$ es bipartito.
Además, el problema pertenece a co-$\mathcal{NP}$ ya que un grafo es 
bipartito si y solo si no tiene ciclos impares, y cualquier ciclo, es un 
certificado suficiente de que el grafo no es bipartito. Por lo tanto, 
este problema pertenece a $\mathcal{NP} \cap$ co-$\mathcal{NP}$.

Consideremos ahora el problema siguiente problema: Dado un grafo $G$ tal 
que $G$ tiene un ciclo hamiltoniano.
Si la respuesta es 'sí', mostrar un ciclo hamiltoniano cualquiera sirve
de certificado suficiente. Sin embargo, si la respuesta es 'no', no
hay un certificado suficiente conocido. A pesar de que claramente
este problema es de la clase $\mathcal{NP}$, no se ha demostrado
que pertenece a co-$\mathcal{NP}$, y es probable que no
pertenezca a esta clase. Pasa lo mismo con el problema de
decisión para caminos hamiltonianos.

Hemos señalado tres relaciones de inclusión entre las clases
$\mathcal{P}$, $\mathcal{N}\mathcal{P}$ y
co$-\mathcal{N}\mathcal{P}$, y es natural preguntarse
si estas inclusiones son adecuadas. Debido a que
$\mathcal{P} = \mathcal{N}\mathcal{P}$
si y solo si $\mathcal{P} = $ co-$\mathcal{N}\mathcal{P}$,
surgen dos preguntas fundamentales, ambas de las cuales se han 
planteado como conjeturas. 

\setlength{\fboxsep}{10pt}
\noindent\fbox{\begin{minipage}{\dimexpr\textwidth-2\fboxsep-2\fboxrule\relax}
 \subsection*{La conjetura de Cook-Edmonds-Levin}
        \begin{align*}
            \mathcal{P}  \neq  \mathcal{N}\mathcal{P}
        \end{align*}
Esta conjetura es una de las más famosas e importantes preguntas abiertas que existen dentro del mundo de la matemática. Fue planteada a mediados de la década de 1960 por J. Edmonds, cuando afirmó que no podría existir un algoritmo 'bueno' (es decir, de tiempo polinómico) para el Problema del Viajante y dio lugar a la definición formal de la Clase \textit{$\mathcal{N}\mathcal{P}$} de Cook(1971) y Levin (1973).
\end{minipage}}

\noindent\fbox{\begin{minipage}{\dimexpr\textwidth-2\fboxsep-2\fboxrule\relax}
 \subsection*{La conjetura de Edmonds}
        \begin{align*}
            \mathcal{P} = \mathcal{NP}
            \cap co-\mathcal{N}\mathcal{P}
        \end{align*}
\indent Esta conjetura también fue propuesta por Edmonds en 1965 y está
fuertemente respaldada por evidencia empírica. Se sabe que la mayoría
de los problemas que sabemos que pertenecen a \newline
\textit{$\mathcal{N}\mathcal{P}$} $\cap$ \textit{co-}$\mathcal{N}\mathcal{P}$
también se sabe que pertenecen a $\mathcal{P}$

\end{minipage}}

\section*{Reducciones Polinomiales}
\indent Es común que para resolver un problema se intente transformar
el problema original en uno cuya solución ya es conocida, para luego
tomar esa solución y transformarla en una para el problema original.
Por supuesto, esto es conveniente solo si la transformación puede
ser hecha rápidamente. Tomando esta idea, definimos a la
\textit{reducción polinomial} de un problema $P$ a un problema $Q$ como el par de algoritmos polinómicos tal que uno transforma cada instancia $I$ de $P$ en una instancia $J$ de $Q$, y el otro transforma la solución para una instancia $Q$ en una solución de una instancia $I$. Si existen estos algoritmos, diremos que la reducción existe y que $P$ es \textit{polinómicamente reducible} a $Q$ y lo representamos como $\textit{P} \preceq \textit{Q}$; esta relación claramente es reflexiva y transitiva.\newline
\indent Una de las claves de la reducción polinomial es que si $\textit{P} \preceq \textit{Q}$ y existe un algoritmo polinómico que resuelve $Q$, entonces ese algoritmo puede ser convertido en un algoritmo polinómico para resolver $P$. En símbolos:
\begin{align*}
    P \preceq Q \land Q \in \mathcal{P} \implies P \in \mathcal{P}
\end{align*}

Un ejemplo de lo mencionado arriba es la reducción polinomial para el
problema de encontrar un árbol recubridor de peso máximo, el problema
consiste en lo siguiente, dado un grafo ponderado conexo $G$, encontrar
un árbol recubridor de peso máximo para el mismo.

Para resolver una instancia del problema mencionado anteriormente,
es suficiente reemplazar cada peso por su opuesto y aplicar el algoritmo
de Prim para encontrar un árbol óptimo en el grafo ponderado resultante.
Este árbol será de peso máximo en el grafo ponderado original.

\section*{La clase $\mathcal{NPC}$ - Problemas $\mathcal{NP}$-Completos}
Podemos definir informalmente a los problemas que pertenecen a la clase
$\mathcal{NPC}$ como los problemas que pertenecen a la clase $\mathcal{NP}$
que son por lo menos tan complicados de resolver como cualquier problema
$\mathcal{NP}$. Formalmente, decimos que un problema P $\in \mathcal{NP}$
es $\mathcal{NPC}$ si $P' \preceq P$ para todo problema P' $\in \mathcal{NP}$.
La clase de problemas $\mathcal{NP-}completos$ es denotada por $\mathcal{NPC}$.

Para probar que un problema Q en $\mathcal{NP}$ es $\mathcal{NPC}$,
es suficiente con encontrar una reducción polinomial para Q a algún P
$\in \mathcal{NPC}$ conocido. En símbolos:
\begin{align*}
    P \preceq Q \land P \in \mathcal{NPC} \implies Q\in   \mathcal{NPC}
\end{align*}

Cook y Levin probaron que en efecto existen problemas $\mathcal{NPC}$.
Más precisamente, probaron que el problema de satisfactibilidad para
fórmulas booleanas es $\mathcal{NPC}$. Describiremos este problema.

\subsection{Satisfactibilidad de fórmula booleana}
Una fórmula booleana es \textit{satisfactible} si hay una asignación
de valores de verdad de sus variables para el cual el valor de
verdad de las formulas es 1. En este casom devimos que la fórmula
es satisfactible por la asignación.

\begin{problem}[Satisfactibilidad Booleana (SAT)]
Dada una formula $f$ nos preguntamos si $f$ es satisfactible.

Observemos que SAT pertenece a $\mathcal{NP}$: Dada una asignación
apropiada de valores de verdad para las variables de $f$, puede
ser verificado en tiempo polinomial que el valor de la fórmula
es en efecto 1. Estos valores de las variables constituyen
un certificado suficiente. Además Cook y Levin probaron que
SAT es un problema $\mathcal{NPC}$.
\end{problem}

\begin{theorem}[El teorema de Cook-Levin]
SAT es un problema de la clase $\mathcal{NPC}$.
\end{theorem}
La prueba del teorema de Cook-Levin utiliza máquinas de Turing
y va más allá de este informe. La prueba puede ser encontrada
en Garey and Johnson (1979) o Sipser (2005).

Aplicando el teorema 1, Karp probó que muchos problemas de
combinaroria son $\mathcal{NPC}$, uno de estos es el problema
de grafos hamililtonianos dirijidos. Para probar estas ideas
necesitamos algunas definiciones.

Una variable $x$, o su negación $\overline{x}$, son literales,
y una disyunción o conjunción de literales es una cláusula
de disyunción o conjunción.

Cualquier conjunción de cláusulas disyuntivas es llamada como
una fórmula en forma normal conjuntiva. Puede ser probado 
(por reducción polinomial) que toda fórmula es equivalente
a una en forma normal conjuntiva. Además, como explicamos más
abajo en la prueba del teorema 2, cada fórmula booleana en 
forma norma conjuntiva es equivalente a una en forma normal
conjuntiva con exactamente tres literales por cláusula. El
problema de decisón para tal fórmula booleana es conocido como
3-SAT

\begin{problem}[Satisfactibilidad Booleana (SAT)]
Dada una fórmula $f$ en forma normal conjuntiva con tres
literales por cláusula, decidir si $f$ es satisfactible.
\end{problem}

\begin{theorem}[Satisfactibilidad Booleana (SAT)]
El problema 3-SAT es $\mathcal{NPC}$.
\end{theorem}

\begin{proof}
Por el teorema 1
es suficiente probar que SAT $\preceq$ 3-SAT. Sea $f$ una
fórmula booleana satisfactible en forma normal conjuntiva.
Mostraremos cómo construir, en tiempo polinómico, una
fórmula booleana $f'$ en forma normal conjuntiva tal que:
\begin{enumerate}
\item Cada cláusula en $f'$ tiene tres literales
\item $f$ es satisfactible si y sólo si $f'$ es satisfactible
\end{enumerate}
\noindent Tal formula $f'$ puede ser obtenida por la adición
de nuevas variables y cláusulas como sigue.

Supongamos que alguna cláusula de $f$ tiene sólo dos literales,
por ejemplo, la cláusula $(x_1 \vee x_2)$. En este caso,
simplemente reemplazamos esta cláusula por dos cláusulas con
tres literales $(x_1 \vee x_2 \vee x)$ y 
$(\overline{x} \vee x_1 \vee x_2)$, donde $x$ es una variable
nueva. Claramente,
\begin{align*}
    (x_1 \vee x_2) \equiv (x_1 \vee x_2 \vee x) \land
    (\overline{x} \vee x_1 \vee x_2)
\end{align*}
Cláusulas con sólo un literal pueden ser tratadas de manera similar.

Ahora supongamos que alguna clausula 
$(x_1 \vee x_2 \vee ... \vee x_k)$ de $f$ tiene $k$ literales, donde
$k \geq 4$. En este caso, agregamos $k-3$ nuevas variables
$y_1,y_2,...,y_{k-3}$ y formamos las siguientes $k-2$ clausulas,
donde cada una tiene tres literales.
\begin{align*}
    (x_1 \vee x_2 \vee y_1), (\overline{y}_1 \vee x_3 \vee y_2),
    (\overline{y}_2 \vee x_4 \vee y_3),..., 
    (\overline{y}_{k-4} \vee x_{k-2} \vee y_{k-3}),
    (\overline{y}_{k-3} \vee x_{k-1} \vee x_k)
\end{align*}
Se puede verificar que $(x_1 \vee x_2 \vee ... \vee x_k)$ es
equivalente a lo conjunción de estas $k-2$ cláusulas.
\end{proof}

El teorema 2 puede ser usado para establecer que varios
problemas en teoría de grafos son $\mathcal{NPC}$, en particular
para el problema de \textit{ciclos hamiltonianos dirigidos}
usando además reducción polinomial.

Como hemos observado, en orden de mostrar que un problema
de decisión $Q$ en $\mathcal{NP}$ es $\mathcal{NP}$, es 
suficiente con encontrar una reducción polinomial para $Q$ de un
problema que se sabe que es $\mathcal{NPC}$.

Veamos un ejemplo de un problema para el cual no se sabe si
pertenece a la clase $\mathcal{P}$ o $\mathcal{NP}$:

\begin{problem}[Isomorfismo de grafos]
Dados dos grafos $G$ y $H$. ¿Son isomorfos?.

La complejidad de este problema es un misterio. Es claro que
este problema pertenece a $\mathcal{NP}$, pero no se sabe
si pertenece a $\mathcal{P}$, $co-\mathcal{NP}$ o a $\mathcal{NPC}$.
Se han encontrado algunos algoritmos polinomiales para algunas
clases de grafos, pero estos algoritmos no sirven para todos
los grafos. Este problema podría ser un contraejemplo para
la conjetura de Edmonds.
\end{problem}

\section*{Los problemas $\mathcal{NP}$-Hard ($\mathcal{NPH}$)}
Así como definimos a los problemas $\mathcal{NP}$-Completo, como aquellos problemas al menos tan complejos como cualquier problema $\mathcal{NP}$, los problemas $\mathcal{NP}$-Hard serán aquellos problemas al menos tan complejos como los $\mathcal{NP}$-Completos, pero con la salvedad de que no tiene por qué ser problemas de decisión.
Al igual que con los $\mathcal{NP}$-Completos, una definición más precisa de los problemas $\mathcal{NPH}$ es la siguiente:
\begin{align*}
    P \preceq Q \land P \in \mathcal{NPH} \implies Q\in   \mathcal{NPH}
\end{align*}
Pero lo interesante sobre los problemas $\mathcal{NPH}$ es lo siguiente: sabemos que cualquier problema $\mathcal{NPC}$ puede ser reducido polinómicamente a otro problema $\mathcal{NPC}$, ahora si encontráramos una solución polinómica a un problema $\mathcal{NPH}$, que por definición es más complejo que cualquier otro problema $\mathcal{NPC}$, entonces podríamos asegurar que existiría al menos una solución polinómica para todos los problemas $\mathcal{NPC}$ y por consiguiente para todos los $\mathcal{NP}$, lo que daría por cierta la conjetura de Edmonds.
Entre los problemas $\mathcal{NPH}$ más conocidos tenemos: dado un grafo $G$, encontrar su clique máxima, encontrar su estable maximo y/o el problema del viajero.

\section*{Aproximación Algorítmica}
Existen ciertos problemas $\mathcal{NPH}$, como el problema del viajero, cuya optimización es de alto interés práctico, pero como vimos anteriormente, todo parece indicar que no va a ser posible encontrar una solución en tiempo polinómica a este problema, por eso es habitual que se busquen de algoritmos que devuelvan resultados lo más cercanos posible al esperado.\newline
Dado un número real t $\geq$ 1, una t-\textit{aproximación algorítmica} de un problema $P$, es un algoritmo que acepta cualquier instancia del problema como input y devuelve una solución fehaciente, cuyo valor no es más que t veces el valor óptimo esperado. Mientras menor la t, mejor será la aproximación.

\end{document}

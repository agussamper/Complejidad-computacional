\documentclass{article}
\usepackage[spanish]{babel}
\usepackage{graphicx} % Required for inserting images
\usepackage{lipsum}
\usepackage[a4paper, total={6in, 8in}]{geometry}
\usepackage{setspace}
\usepackage[utf8]{inputenc}
\usepackage{csquotes}

\setlength{\parskip}{2pt}

\spacing{1.3}

\title{Complejidad computacional}
\author{Agustín Samper, Facundo Llaudet y Guillermo Pereyra}
\date{November 2023}

\begin{document}
\maketitle
\section*{Introducción}
%\setlength{\parindent}{15pt}
Veremos como los problemas pueden ser clasificados por su nivel de dificultad. \newline
\indent La mayoría de los problemas que consideramos en la materia son de 
carácter general, aplicados a todos los miembros de alguna familia de
grafos o digrafos. Por una instancia de un problema, nos referimos al
problema aplicado a un miembro específico de la familia. Por ejemplo,
una instancia del problema de el árbol recubridor de peso mínimo es el
problema de encontrar un árbol óptimo (árbol recubridor de peso mínimo)
para un grafo ponderado particular. \newline
%\setlength{\parindent}{15pt}
\indent Un algoritmo para resolver un problema es un procedimiento
computacional bien definido, el cual acepta cualquier instancia del problema
como entrada y retorna una solución al problema como salida. Por ejemplo,
el algoritmo de Prim acepta como entrada un grafo ponderado y retorna un
árbol óptimo. \newline
%ver si agregar lo que está antes en el libro
\indent Por complejidad computacional de un algoritmo, nos referimos a el
número de pasos básicos computacionales (como operaciones aritméticas y
comparaciones) requeridas para su ejecución. Este número depende claramente
de la tamaño y de la naturaleza de la entrada. \newline
\indent Diremos que un algoritmo es polinomial cuando el número de
operaciones que efectúa está acotado por una función polinomial en
el tamaño de su entrada, el algoritmo es llamado \textit{algoritmo
de tiempo polinomial}. Además, está calificado como \textit{tiempo lineal}
si el polinomio es una función lineal, \textit{tiempo cuadrático} si 
es una función cuadrática y así sucesivamente.

\section*{La clase \textit{P}}
La importancia de algoritmos de tiempo polinomial es que estos algoritmos
generalmente son computacionalmente realizables, incluso para grandes
tamaños de entrada. En contraste, los algoritmos cuya complejidad es
exponencial en el tamaño de su entrada tienen tiempos de ejecución los
cuales los hacen inutilizables, incluso para entradas de un tamaño
moderado.

\end{document}
